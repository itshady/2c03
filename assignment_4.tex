\PassOptionsToPackage{table}{xcolor}
\documentclass{article}
\usepackage[colorlinks]{hyperref}
\usepackage[margin=1.25in]{geometry}
\usepackage{amsmath,amssymb,amsthm,booktabs,tikz}
\usepackage[final]{microtype}
\usepackage{libertine}
\usepackage[varqu]{zi4}
\usepackage[libertine]{newtxmath}
\usepackage[T1]{fontenc}
\usepackage[utf8]{inputenc}
\usepackage{tabto}
\usepackage[normalem]{ulem}

\usepackage{amsmath,amssymb,amsthm,booktabs,tikz}
\usepackage[final]{microtype}
\usepackage{libertine}
\usepackage[varqu]{zi4}
\usepackage[libertine]{newtxmath}
\usepackage[T1]{fontenc}
\usepackage[utf8]{inputenc}
\usepackage{multirow,tabto,siunitx}

\usetikzlibrary{shapes.multipart}

\usetikzlibrary{arrows.meta,calc,decorations.pathreplacing,shapes.geometric,shapes.multipart,overlay-beamer-styles}
\tikzset{
    >=Stealth,
    dot/.style={circle,scale=0.25,draw=black,fill=black},
    every node/.append style={align=center,font=\strut},
    every node part/.append style={align=center},
    lnode/.style={rectangle split,rectangle split parts=2,font=\strut,rectangle split part fill={black!10,cbYellow!10},rectangle split part align={center}},
    hlisted/.style={right,rectangle split,rectangle split,draw,rectangle split parts=#1,font=\strut,rectangle split part fill={black!10},rectangle split part align={center},text width=0.5cm},
}

\newcommand{\At}[1]{\texttt{@#1}}
\newcommand{\Null}{\texttt{@null}}


% Seven colors safe for use color blindness.
% Colors taken from doi:10.1038/nmeth.1618.
\definecolor{cbOrange}{RGB}{230,159,0}
\definecolor{cbSkyBlue}{RGB}{86,180,233}
\definecolor{cbBluishGreen}{RGB}{0,158,115}
\definecolor{cbBlue}{RGB}{0,114,178}
\definecolor{cbVermillion}{RGB}{213,94,0}
\definecolor{cbReddischPurple}{RGB}{204,121,167}
\definecolor{cbYellow}{RGB}{240,228,66}

\theoremstyle{definition}
\newtheorem{problem}{Problem}
\newenvironment{questions}{\begin{enumerate}
\renewcommand{\theenumi}{P\arabic{problem}.\arabic{enumi}}}{\end{enumerate}}
\newcommand{\HL}[1]{\textcolor{cbReddischPurple}{#1}}

\newcommand{\abs}[1]{\lvert #1 \rvert}
\newcommand{\OOG}[1]{\mathord{\sim}#1}
\DeclareMathOperator{\bdiv}{div}

\newcommand{\BigO}[1]{\mathcal{O}\left(#1\right)}
\newcommand{\BigOmega}[1]{\Omega\left(#1\right)}
\newcommand{\BigTheta}[1]{\Theta\left(#1\right)}


\newcommand{\True}{\texttt{true}}
\newcommand{\False}{\texttt{false}}

%% Tikz
\usepackage{tikz}
\usetikzlibrary{arrows.meta,calc,decorations.pathreplacing,shapes.geometric,shapes.multipart,overlay-beamer-styles}
\tikzset{
    >=Stealth,
    dot/.style={circle,scale=0.35,draw=black,fill=black},
    stacked/.style={above,rectangle split,draw,rectangle split parts=#1,font=\strut,rectangle split part fill={none,black!10}},
    centered/.append style={align=center}
}


% Algorithm
\usepackage{algorithmic}
\newcommand{\GETS}{:=}
\newcommand{\VAR}[1]{\textit{#1\/}}
\newcommand{\AName}[1]{\textsc{#1}}
\renewcommand{\algorithmicrequire}{\textbf{Input:}}
\renewcommand{\algorithmicensure}{\textbf{Result:}}
\newcommand{\CMT}[1]{\text{``#1''}}
\renewcommand{\algorithmiccomment}[1]{\CMT{#1}}
\newcommand{\INV}[1]{\emph{inv: } #1}
\newcommand{\VF}[1]{\emph{bf: } #1}
\makeatletter
\newlength{\Algo@MyTabLength}
\newcommand{\AlgoTabTo}[1]{%
    \setlength{\Algo@MyTabLength}{#1}%
    \addtolength{\Algo@MyTabLength}{-\ALC@tlm}%
    \tabto{\Algo@MyTabLength}%
}
\newenvironment{myonlyalgo}[1][0]{
    \vskip 5pt
    \hrule
    \smallskip
    \begin{algorithmic}[1]
    \setcounter{ALC@line}{#1}
}{
    \end{algorithmic}
    \hrule
    \vskip 5pt
}
\newenvironment{myalgo}[2][0]{
    \vskip 5pt
    \hrule
    \smallskip
    \noindent{\textbf{Algorithm} #2\textbf{:}}
    \begin{algorithmic}[1]
    \setcounter{ALC@line}{#1}
}{
    \end{algorithmic}
    \hrule
    \vskip 5pt
}







%% Metadata
\newcommand{\Assignment}[1]{
    \title{\vskip-2em%The standard article.cls package puts 2em whitespace on top of the title, undo this.
           Assignment #1 - \Large{Hady Ibrahim}\\{\Large SFWRENG 2CO3: Data Structures and Algorithms--Winter 2023}}}
\newcommand{\Deadline}[1]{
    \author{Deadline: #1}
}
\date{{\normalsize
    Department of Computing and Software\\
    McMaster University
}}

\newcommand{\Warning}[1]{\textbf{\textcolor{red!80!black}{#1}}}
\renewcommand{\labelitemi}{$\blacktriangleright$}

\newcommand{\DEFAULTMSG}{
Please read the \emph{Course Outline} for the general policies related to assignments.
\begin{center}
\Warning{Plagiarism is a \underline{\textit{\vphantom{y}serious academic offense}} and will be handled accordingly.}\\
\Warning{All suspicions will be reported to the \underline{\textit{Office of Academic Integrity}}\\(in accordance with the \href{https://secretariat.mcmaster.ca/app/uploads/Academic-Integrity-Policy-1-1.pdf}{Academic Integrity Policy}).}
\end{center}

This assignment is an \emph{individual} assignment: do not submit work of others. All parts of your submission \emph{must} be your own work and be based on your own ideas and conclusions. Only \emph{discuss or share} any parts of your submissions with your TA or instructor.  You are \emph{responsible for protecting} your work: you are strongly advised to password-protect and lock your electronic devices (e.g., laptop) and to not share your logins with partners or friends! If you \emph{submit} work, then you are certifying that you have completed the work for this assignment by yourself. By submitting work, you agree to automated and manual plagiarism checking of all submitted work.

\emph{Late submission policy}. Late submissions will receive a late penalty of 20\% on the score per day late (with a five hour grace period on the first day, e.g., to deal with technical issues) and submissions five days (or more) past the due date are not accepted. In case of technical issues while submitting, contact the instructor \emph{before} the deadline.}

\newcommand{\SUBMITMSG}{\section*{Assignment Details}
Write a report in which you solve each of the above problems. Your submission:
\begin{enumerate}
\item must be a \texttt{PDF} file;
\item must have clearly labeled solutions to each of the stated problems;
\item must be clearly presented;
\item must \emph{not} be hand-written: prepare your report in \LaTeX{} or in a word processor such as Microsoft Word (that can print or exported to \texttt{PDF}).
\end{enumerate}
\Warning{Submissions that do not follow the above requirements will get a grade of zero.}}

\newcommand{\DEFAULTGRADING}{\section*{Grading}
Each problem counts equally toward the final grade of this assignment.
}


\Assignment{4}
\Deadline{March 19, 2023}
\begin{document}
\maketitle
\DEFAULTMSG{}

\begin{problem}
Consider a directed graph $\Graph = (\Nodes, \Edges)$. Breadth-first search and depth-first search both have a runtime complexity of $\OOG{\abs{\Nodes} + \abs{\Edges}}$ if we represent the graph with an adjacency list.
\begin{questions}
\item What is the runtime complexity of breadth-first search and depth-first search when we use the matrix representation? Explain your answer.
\item Consider the \emph{ordered edge list} representation that represents a graph by an \emph{ordered} array $A$ that holds $\abs{\Edges}$ edges. The order of edges is as follows: edge $(n_1, m_1) \in A$ comes before $(n_2, m_2) \in A$ if $\ID{n_1} < \ID{n_2}$ or if $\ID{n_1} = \ID{n_1} \land \ID{m_1} \leq \ID{m_2}$. What is the runtime complexity of breadth-first search and depth-first search when we use the \emph{ordered edge list} representation? Explain your answer.
\end{questions}
\end{problem}

\begin{problem}
Consider a directed acyclic graph $\Graph = (\Nodes, \Edges)$ with a source node $m \in \Nodes$ and target node $n \in \Nodes$.
\begin{questions}
\item Provide an algorithm that computes the number of paths starting at $m$ and ending at $n$.

Assuming an adjacency representation of the graph.

\begin{myalgo}{\AName{DFS}($\VAR{start},\VAR{dest},\VAR{storedPaths}$)}
  \IF{$\VAR{start}=1$}
    \RETURN 1
  \ENDIF
  \STATE $path = 0$
  \FOR{$n \in \VAR{adj[start]}$}
    \IF{$(storedPaths.containsKey(n))$}
      \STATE $path += storedPaths.get(n)$
    \ELSE{}
      \STATE $\VAR{paths} += $\AName{DFS}($\VAR{start},\VAR{dest},\VAR{storedPaths}$)
    \ENDIF
  \ENDFOR
  \STATE $storedPaths.putIfAbsent(v, path)$
  \RETURN $paths$
\end{myalgo}

Call this via 

\begin{myalgo}{\AName{Main}()}
  \STATE $graph \GETS new Graph()$  \COMMENT{initialized graph with edges and nodes already set}
  \STATE $Map<Integer, Integer> storedPaths = new HashMap<>()$
  \PRINT \AName{DFS}($start\_node\_id,destination\_node\_id,storedPaths$)
\end{myalgo}

Graph representation: adjacency representation. The algorithm runs DFS from the start node. For each node in its adjacency list (neighbours), it runs another DFS. This happens recursively.
This algorithm goes over what is considered a "marked node" in normal DFS as we want to account for every single path, not only unique ones. Each DFS counts the number of paths to the destination node by summing the number of paths to the destination node all of its children have.
The base case is when reaching the destination node, we have 1 path. Using this recursion, we can sum all the paths to the destination node and return that.

As I use dynamic programming to keep track of the solution to "subproblems" (a subproblem being finding the paths to the destination node from specific node), we only ever need to compute the number of paths from a single node to the destination once. After that we can fetch that result from the storedPaths map.
This means that for every node, we perform a DFS. We know DFS using adjacency represenation is $\OOG{|N| + |E|}$, so if we do this for each node, the complexity is $\OOG{|N|^2 + |N||E|}$

\item We say that the directed acyclic graph $\Graph$ has a \emph{bottleneck} if there is a node $b$, distinct from source $m$ and target $n$, such that all paths from $m$ to $n$ go through node $b$. Write an algorithm that returns $\True$ if  and only if $\Graph$ has such a bottleneck. You may assume that there is at-least one path from $m$ to $n$.
\end{questions}
For each question, explain why your algorithm is correct, what the complexity of your algorithm is, and which graph representation you use.
\end{problem}

\begin{problem}
Let $n$ be a positive integer and consider two $n \times n$ matrices $M_1$ and $M_2$. The \emph{Boolean matrix product} of $M_1$ and $M_2$, denoted by $M' \GETS M_1 \otimes M_2$, is the $n \times n$ matrix $M'$ in which $M'[i, j]$ is:
\[ M'[i, j] = (M_1[i, 0] \land M_2[0, j]) \lor (M_1[i, 1] \land M_2[1, j]) \lor \dots \lor (M_1[i, n-1] \land M_2[n-1,j]). \]

Now consider a directed graph $\Graph = (\Nodes, \Edges)$ implemented via the matrix representation $M$ with $n = \abs{\Nodes}$.
\begin{questions}
\item Let $M' = M \otimes M$. What does the value $M'[i,j]$ represent (when is it $\True$ and when is it $\False$)?
\item Consider paths of length $k$ in graph $G$. Provide an algorithm that uses the Boolean matrix product operations to compute a matrix $M_k$ such that, for every pair of nodes $n, m \in \Nodes$, $M_k[\ID{n}, \ID{m}]$ is $\True$ if and only if there is a path of length exact-$k$ from node $n$ to node $m$. Explain why your algorithm is correct and provide the complexity of your algorithm in terms of the number of Boolean matrix product operations used.

\textit{Hint}. One can design an algorithm that performs at-most $\OOG{\log_2(k)}$ Boolean matrix product operations.
\item The \emph{transitive closure} of $\Graph$ is a graph $\Graph' = (\Nodes, \Edges')$ such that $(n, m) \in \Edges'$ if and only if there is a path from node $n$ to node $m$ in graph $\Graph$. Provide an algorithm that uses the Boolean matrix product operations to compute a matrix $M'$ that represents graph $\Graph'$. Explain why your algorithm is correct and provide the complexity of your algorithm.
\end{questions}
\end{problem}

\begin{problem}
Consider a directed graph $\Graph = (\Nodes, \Edges)$ and assume we have a weight function $\Weight : \Edges \rightarrow \{1, \dots, W\}$ with $W$ some maximum integer value $W$.
\begin{questions}
\item Assume $W = 1$ (all edges have weight $1$). Given node $n \in \Nodes$, provide an algorithm that can compute the shortest path from node $n$ to any other node $m$ in $\OOG{\abs{\Nodes} + \abs{\Edges}}$. In this case, the shortest path is the path with the fewest edges.
\item Given node $n \in \Nodes$, provide an algorithm that can compute the shortest path (in terms of the sum of the weights of edges on the path) from node $n$ to any other node $m$ in $\OOG{\abs{\Nodes} + W\abs{\Edges}}$.
\item Given node $n \in \Nodes$, provide an algorithm that can compute the shortest path (in terms of the sum of the weights of edges on the path) from node $n$ to any other node $m$ in $\OOG{W\abs{\Nodes} + \abs{\Edges}}$.
\end{questions}
For each question, explain why your algorithm is correct, why your algorithm achieves the stated complexity, and which graph representation you use.
\end{problem}

\SUBMITMSG{}
\DEFAULTGRADING{}

\end{document}